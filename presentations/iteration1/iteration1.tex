\documentclass[mathserif,serif]{beamer}

\usepackage{graphicx}
\shorthandoff{=}

%\setbeamertemplate{blocks}[rounded][shadow=true]
\setbeamertemplate{background canvas}[vertical shading][bottom=white,top=structure.fg!25]

\mode<presentation>

\newcommand {\framedgraphic}[2] {
    \begin{frame}{#1}
        \begin{center}
            \includegraphics[width=\textwidth,height=0.8\textheight,keepaspectratio]{#2}
        \end{center}
    \end{frame}
}


\title{SWOP iteratie 1}
\author{Pablo en Co (groep 2)}
\institute{KU Leuven}
\date{2015}

\begin{document}
 
  \frame{\titlepage}
  \framedgraphic{High Level}{../../diagrams/high_level_uml.eps}
  \begin{frame}{Interessant: Facadecontroller}
      \begin{center}
      \includegraphics[width=\textwidth,height=0.5\textheight,keepaspectratio]{../../diagrams/facade_uml.eps}
        \begin{itemize}
        \item Duidelijke verantwoordelijkheid
        \item Wrapping en unwrapping
        \item Bescherming van domain
        \end{itemize}
      \end{center}
  \end{frame} 
  \begin{frame}{Interessant: Class Diagram}
      \begin{center}
      \includegraphics[width=\textwidth,height=0.7\textheight,keepaspectratio]{../../diagrams/domain_uml.eps}
        \begin{itemize}
        \item Status beslist over functionaliteit van Taak.
        \item Task beslist welke zijn status is.
        \end{itemize}
      \end{center}
  \end{frame} 
  \framedgraphic{Task DFS}{../../diagrams/task_dfs.eps}
  \begin{frame}{Interessant: Project}
      \begin{center}
      \includegraphics[width=\textwidth,height=0.5\textheight,keepaspectratio]{../../diagrams/project_dfs.eps}
        \begin{itemize}
        \item Status hangt enkel af van taken ...
        \item ... en geen specifieke functionaliteit per status.
        \item $\Rightarrow$ Geen extra klasse nodig
        \end{itemize}
      \end{center}
  \end{frame} 
  \begin{frame}{Interessant: Create Task}
      \begin{center}
        \includegraphics[width=\textwidth,height=0.6\textheight,keepaspectratio]{../../diagrams/sequence_createTask_uml.eps}
        \begin{itemize}
        \item ProjectWrapper
        \item ProjectData   
        \item createTaskFor(projectWrapper, projectData)
        \end{itemize}
      \end{center}
  \end{frame}
  \begin{frame}{Testing}
      \begin{center}
        \begin{itemize}
        \item Unit tests per klasse
        \item Use case tests
        \item Scenario test
        \item Button masher UI
        \end{itemize}
      \end{center}
  \end{frame} 

% etc
\end{document}




%%% Local Variables:
%%% mode: latex
%%% TeX-master: " RET"
%%% End:
